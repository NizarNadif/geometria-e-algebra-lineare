\chapter{Determinante di una matrice}

	Il determinante di una matrice è uno scalare ricavabile dalle matrici quadrate.
	
	Quando tale valore è differente da 0, il rango della matrice è massimo (ed essa è dunque invertibile), altrimenti tale rango è inferiore al massimo \textit{n}.
	
	\paragraph{Definizione} Sia $A \in M_n (\mathbb{R})$. Il \textbf{determinante} di A è definito ricorsivamente:
	\begin{enumerate}
		\item se $n = 1\, , \; \det A = \det [a_{11}] = a_{11} $
		\item se $n \ge 1\, , $  
		$$ \det A = \sum_{i=1}^{n} a_{i1} a_{i1}^{\prime} \, , \qquad \text{con } a_{i1}^{\prime} = (-1)^{i + 1} \det A_{i1} $$
		
		Il determinante è dunque pari alla somma dei prodotti fra gli elementi della prima colonna ed il loro \emph{complemento algebrico} ($a_{i1}^{\prime}$). Il complemento algebrico rende tale formula ricorsiva, dato che questo ultimo è pari prodotto fra $(-1)^{i + 1}$ (che fa variare il segno a seconda della riga nella quale si trova il numero preso in esame) ed il determinante della matrice $A_{i1}$, ottenuta a seguito della rimozione della prima colonna e della \textit{i}-esima riga dalla matrice presa in esame.
	\end{enumerate}
	
	\section{Proprietà}
		\begin{enumerate}[(1)]
			\item Se \textit{B} è ottenuta dallo scambio di due righe di \textit{A}, il suo determinante è l'opposto di quello della matrice originale.
			$$ \det (S_{ij} A) = - \det A $$
			\item Se \textit{B} è ottenuta da \textit{A} a seguito della moltiplicazione di una riga con uno scalare \textit{c}, il suo determinante è pari al rapporto fra tale scalare ed il determinante della matrice originale.
			$$ \det (D_i (c) A) = c \cdot \det A $$
			\item Se \textit{B} è ottenuta da \textit{A} in seguito alla somma fra una riga ed il multiplo di un'altra, il suo determinante è uguale a quello della matrice originale.
			$$ \det (E_{ij}(c)A) = \det A $$
			\item se la matrice ha due righe uguali o una riga nulla, ed ha dunque un rango non massimo, il suo determinante \underline{è pari a 0}.
			\item dato $k \in \mathbb{K}$, il determinante del prodotto fra una matrice e tale scalare è pari al prodotto fra $k^n$ (dove \textit{n} è la dimensione della matrice) ed il determinante della matrice. Ciò vale grazie alla proprietà (2), dato che tale scalare viene moltiplicato per tutte le \textit{n} righe.
			$$ \det (k \cdot A) = k^n \det A $$
		\end{enumerate}
		
		Quando la matrice presa in esame è una matrice identica, le prime tre proprietà sono le seguenti:
		\begin{enumerate}[(1')]
			\item $\det (S_{ij}) = -1 $
			\item $\det (D_i (c)) = c $
			\item $\det (E_{ij} (c)) = 1 $
		\end{enumerate}
	
	\section{Teorema di Binet}
		Date due matrici quadrate con le stesse dimensioni, il determinante del loro prodotto matriciale è pari al prodotto fra i determinanti delle due matrici.
		$$ \det (A B) = (\det A) \cdot (\det B) \: , \qquad \forall A, B \in M_n$$
	
		\subsection{Corollario}
			mediante il precedente teorema è possibile affermare che il determinante di una matrice inversa è uguale all'inverso del determinante della matrice:
		$$ \det (A^{-1}) = \frac{1}{\det A} $$
	
	\section{Teorema della matrice trasposta}
		Il determinante di una matrice è pari al determinante della sua matrice trasposta:
		$$ \det (A^T) = \det A $$
		Questo teorema è importante perché ci permette di estendere alle colonne l'uso delle proprietà precedentemente enunciate.
	
	\section{Matrice a scalini} 
		Il determinante di una matrice è diverso da zero se e solo se lo è anche quello di una matrice a scalini \textit{S} ricavata dalla matrice; in tal caso il rango della matrice è massimo e dunque la matrice è invertibile.
		$$ \det A \neq 0 \iff \det S \neq 0 \iff \text{rg} A = n \iff A \text{ è invertibile}  $$
		
		\begin{GrayBox}
			\paragraph{Dimostrazione}
			è possibile ricavare una matrice a scalini \textit{S} a seguito di una serie di operazioni elementari
			$$ S = E_k \dots E_1 A $$
			dunque, per il teorema di Binet, il suo determinante è
			$$ \det S = (\det E_k) \dots (\det E_1) (\det A) $$
			Per le proprietà (1'), (2') e (3') è possibile dire che il determinante delle matrici elementari non è mai nullo e dunque, è possibile dire che affinché il determinante di \textit{S} sia pari a 0, il determinante di \textit{A} lo deve essere.
		\end{GrayBox}
	
	\section{Sviluppi di Laplace}
		Sia $1 \leq j \leq n$ un numero finito,
		$$ \det A = \sum_{i = 1}^{n} a_{ij} a_{ij}^{\prime} \, , \quad \text{con } a_{ij}^{\prime} = (-1)^{i + j} \det A_{ij} $$
		Sia $1 \leq i \leq n$ un numero finito,
		$$ \det A = \sum_{j = 1}^{n} a_{ij} a_{ij}^{\prime} $$
		
		\begin{GrayBox}
			\paragraph{Esempio} Sviluppiamo la seconda riga
			$$ 
			\det \begin{bmatrix}
				3 & 2 & 2 \\
				3 & 0 & 1 \\
				3 & 1 & 2
			\end{bmatrix}
			= 3 \left(- \det \begin{bmatrix}
				2 & 2 \\
				1 & 2
			\end{bmatrix} \right)
			+ 1 \left(- \det \begin{bmatrix}
				3 & 2 \\
				3 & 1
			\end{bmatrix} \right)
			= -3 \cdot (4 - 2) - 1 \cdot (3 - 6) = -3
			$$
		\end{GrayBox}
	
		\subsection{Aree e volumi}
			Mediante gli sviluppi di Laplace è possibile descrivere pure l'area di un parallelogrammo generato da due vettori $\vec{v}, \vec{w} \in V^2$ come combinazione:
			
			Dati $\vec{v} = (v_1, v_2)$ e $\vec{w} = (w_1, w_2)$ non proporzionali, sappiamo che l'aera del parallelogrammo generato da essi è la seguente:
			$$A^2 = (v_1 w_2 - v_2 w_1)^2$$
			Possiamo notare che il contenuto di tale quadrato può essere descritto come determinante di una matrice quadrata di dimensione 2:
			$$A^2 = \left( \det \begin{bmatrix}
				v_1 & v_2 \\
				w_1 & w_2
			\end{bmatrix}\right)^2$$
			da ciò ne deriva che
			$$A = \left\vert \det \begin{bmatrix}
				v_1 & v_2 \\
				w_1 & w_2
			\end{bmatrix}\right\vert$$
			
			Spostiamoci in $V^3$ ed aggiungiamo alla matrice una prima riga nella quale sono presenti le direzioni degli assi vettoriali e la terza componente di entrambi i vettori, sviluppando la prima riga otteniamo la somma dei prodotto fra i vettori direzione degli assi ed i componenti del prodotto vettoriale:
			$$\det \begin{bmatrix}
				\vec{i} & \vec{j} & \vec{k} \\
				v_1 & v_2 & v_3 \\
				w_1 & w_2 & w_3
			\end{bmatrix} = \vec{i} \cdot (v_2 w_3 - v_3 w_2) + \vec{j} \cdot (v_3 w_1 - v_1 w_3) + \vec{k} \cdot (v_1 w_2 - v_2 w_1) = \vec{v} \times \vec{w} 
			$$
			
			Sapendo che il volume di un parallelepipedo è pari al modulo del prodotto misto fra i 3 vettori $\vec{u}, \vec{v}, \vec{w} \in V^3$ che lo descrivono, è necessario sostituire le componenti di $\vec{u}$ all'interno della precedente matrice per ottenere la stessa formula:
			$$V = \vert \vec{u} \cdot (\vec{v} \times \vec{w}) \vert = 
			\left\vert \det \begin{bmatrix}
				u_1 & u_2 & u_3 \\
				v_1 & v_2 & v_3 \\
				w_1 & w_2 & w_3
			\end{bmatrix} \right\vert$$
	\section{Sistemi lineari}
		Dato un sistema $\underset{n \times n}{Ax} = b$, quest'ultimo ha un'unica soluzione $x = A^{-1}b$ se il determinante della matrice A è diverso da zero.
		$$\det A \neq 0 \implies \exists x = A^{-1} b$$
		
		\subsection{Regola di Cramer}
			Sia $A_j(b)$ la matrice ottenuta da \textit{A} sostituendo \textit{b} alla \textit{j}-esima colonna, per il teorema di Binet si ha			
			$$\det (A^{-1} A_j (b)) = \frac{\det A_j (b)}{\det A} = \det (A^{-1} \underbrace{(A^1 \dots b \dots A^n)}_{\text{colonne}}) = \det (e_1 \dots x \dots e_n) = x_j$$
			$$x_j = \frac{\det A_j (b)}{\det A} \: , \quad \forall j = 1, \dots, n$$
			Si ottiene così la regola di Cramel mediante la quale è possibile trovare tutte le componenti della soluzione $(x_1, \dots, x_j)$.
		
			\subsubsection{Matrice inversa con Cramer}
			
			