% modifico i margini della pagina (quelli predefiniti sono 1.5in)
\usepackage[margin=1.25in]{geometry}

%codificazione del documento
%codificazione del testo inserito
\usepackage[utf8]{inputenc}
%codificazione del testo stampato
\usepackage[T1]{fontenc}

%scelta della lingua italiana
\usepackage[italian]{babel}

%libreria utilizzata per includere le immagini
\usepackage{graphicx}
%definisco la cartella dalla quale recuperarle
\graphicspath{{assets/}}

%libreria utilizzata per includere i link
\usepackage[hidelinks]{hyperref}

%libreria utilizzata per includere i comment
%/begin{comment} ... /end{comment}
\usepackage{comment}

%libreria  che ci permette di modificare l'header
\usepackage{fancyhdr}

%libreria che mi permette di modificare la numerazione nell'indice
\usepackage{tocloft}
%faccio in modo che lo stile di pagina di questa libreria non sovrascriva
%quello fatto da me con la libreria fancyhdr
\tocloftpagestyle{fancy}

%libreria che rimuove lo spazio all'inizio di ogni paragrafo
\usepackage[parfill]{parskip}

%libreria che mi permette di accostare un'immagine a del testo
\usepackage{wrapfig}

%libreria che permette di aggiungere i commenti
\usepackage{verbatim}

%libreria che include i somboli matematici
\usepackage{amsmath,amsfonts,amssymb,amsthm,mathtools, nicefrac, nicematrix} % AMS

%libreria per includere le box colorate
\usepackage[most]{tcolorbox}

%libreria che permette di stampare dei lorem ipsum
\usepackage{lipsum}

%libreria che permette di modificare gli enumerate
\usepackage{enumerate}

%libreria che permette di creare assets personalizzati
\usepackage{tikz, tikz-3dplot}
\usetikzlibrary{shapes.misc,shadows}

%libreria che permette di modificare lo stile delle titolazioni
\usepackage[ ]{titlesec}